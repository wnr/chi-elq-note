% THIS IS SIGPROC-SP.TEX - VERSION 3.1
% WORKS WITH V3.2SP OF ACM_PROC_ARTICLE-SP.CLS
% APRIL 2009
%
% It is an example file showing how to use the 'acm_proc_article-sp.cls' V3.2SP
% LaTeX2e document class file for Conference Proceedings submissions.
% ----------------------------------------------------------------------------------------------------------------
% This .tex file (and associated .cls V3.2SP) *DOES NOT* produce:
%       1) The Permission Statement
%       2) The Conference (location) Info information
%       3) The Copyright Line with ACM data
%       4) Page numbering
% ---------------------------------------------------------------------------------------------------------------
% It is an example which *does* use the .bib file (from which the .bbl file
% is produced).
% REMEMBER HOWEVER: After having produced the .bbl file,
% and prior to final submission,
% you need to 'insert'  your .bbl file into your source .tex file so as to provide
% ONE 'self-contained' source file.
%
% Questions regarding SIGS should be sent to
% Adrienne Griscti ---> griscti@acm.org
%
% Questions/suggestions regarding the guidelines, .tex and .cls files, etc. to
% Gerald Murray ---> murray@hq.acm.org
%
% For tracking purposes - this is V3.1SP - APRIL 2009

\documentclass{acm_proc_article-sp}

\begin{document}

\title{Allowing Responsive Web Modules}
\subtitle{}
%
% You need the command \numberofauthors to handle the 'placement
% and alignment' of the authors beneath the title.
%
% For aesthetic reasons, we recommend 'three authors at a time'
% i.e. three 'name/affiliation blocks' be placed beneath the title.
%
% NOTE: You are NOT restricted in how many 'rows' of
% "name/affiliations" may appear. We just ask that you restrict
% the number of 'columns' to three.
%
% Because of the available 'opening page real-estate'
% we ask you to refrain from putting more than six authors
% (two rows with three columns) beneath the article title.
% More than six makes the first-page appear very cluttered indeed.
%
% Use the \alignauthor commands to handle the names
% and affiliations for an 'aesthetic maximum' of six authors.
% Add names, affiliations, addresses for
% the seventh etc. author(s) as the argument for the
% \additionalauthors command.
% These 'additional authors' will be output/set for you
% without further effort on your part as the last section in
% the body of your article BEFORE References or any Appendices.

\numberofauthors{3} %  in this sample file, there are a *total*
% of EIGHT authors. SIX appear on the 'first-page' (for formatting
% reasons) and the remaining two appear in the \additionalauthors section.
%
\author{
% You can go ahead and credit any number of authors here,
% e.g. one 'row of three' or two rows (consisting of one row of three
% and a second row of one, two or three).
%
% The command \alignauthor (no curly braces needed) should
% precede each author name, affiliation/snail-mail address and
% e-mail address. Additionally, tag each line of
% affiliation/address with \affaddr, and tag the
% e-mail address with \email.
%
% 1st. author
\alignauthor
  Ben Trovato\\
  \affaddr{Institute for Clarity in Documentation}\\
  \affaddr{1932 Wallamaloo Lane}\\
  \affaddr{Wallamaloo, New Zealand}\\
  \email{trovato@corporation.com}
% 2nd. author
\alignauthor
  G.K.M. Tobin\\
  \affaddr{Institute for Clarity in Documentation}\\
  \affaddr{P.O. Box 1212}\\
  \affaddr{Dublin, Ohio 43017-6221}\\
  \email{webmaster@marysville-ohio.com}
% 3rd. author
\alignauthor
  Lars Th{\o}rv{\"a}ld\\
  \affaddr{The Th{\o}rv{\"a}ld Group}\\
  \affaddr{1 Th{\o}rv{\"a}ld Circle}\\
  \affaddr{Hekla, Iceland}\\
  \email{larst@affiliation.org}
}

\date{10 October 2015}
% Just remember to make sure that the TOTAL number of authors
% is the number that will appear on the first page PLUS the
% number that will appear in the \additionalauthors section.

\maketitle
\begin{abstract}

\end{abstract}

% A category with the (minimum) three required fields
\category{H.4}{Information Systems Applications}{Miscellaneous}
%A category including the fourth, optional field follows...
\category{D.2.8}{Software Engineering}{Metrics}[complexity measures, performance measures]

\terms{Theory}

\keywords{ACM proceedings, \LaTeX, text tagging} % NOT required for Proceedings

\section{Introduction}
  % \begin{itemize}
  %   \item Why modules? Reusability (even across applications), reduced code complexity.
  %   \item Why responsive design?
  %   \item Responsive Modules of today need to be context aware (thus, not very reusable [they only work in a specific layout]).
  %   \item What do we want and why? Modules that are responsive relative to its outer frame.
  % \end{itemize}

  A module is an interchangeable and independent part of a program that typically has a single and well-defined responsibility \cite{parnas1972criteria}.
  Modular programming is a technique to reduce complexity and enable reusability.
  In order for a module to be reusable it must not assume in which context it is being used.

  Responsive Web Design (RWD) is an approach to make the application design respond to the viewport size, in order to support varying devices.
  This is achieved by using CSS media queries to define conditional style rules.

  The problem is that there is no way to make a module responsive without it being context-aware, due to media queries only being able to target the viewport.
  Thus, a responsive module using media queries is layout dependent and has therefore limited reusability.

  The desired behavior of a responsive module is having its inner design responding to the size of \emph{its frame} instead of the viewport.
  Only then is a responsive module independent of its layout context.

  This can be achieved with the theoretical feature \emph{element queries} that enables conditional CSS rules by the properties of arbitrary elements.
  This paper presents a novel implementation of element queries in JavaScript, and discusses the new possibilities of GUI design that our implementation enables.

  \subsection{The Problem Exemplified}
    \begin{itemize}
      \item MQ is not the solution to RWD. (MQ was not designed for RWD as the feature was released long before RWD)
      \item All elements adapt their inner design by the viewport width.
      \item Menu Example shows how MQ are broken.
      \item Limitations of MQ regarding font-size (em).
    \end{itemize}

    Media queries were designed to enable developers to conditionally design content by the media, such as using serif fonts when printed and sans-serif when viewed on a screen.
    Therefore, it is only applicable for RWD of non-modular static applications.
    In a world where no better solution than media queries exists for RWD, changing the layout of a responsive application becomes a cumbersome task.
    
    Imagine an application that displays the current weather of various cities as widgets, by using a weather widget module.
    The module should be responsive, so that more information such as a temperature graph over time is displayed when the widget is big.
    When the widget is small it should only display the current temperature.
    Users should also be able to add, remove and resize widgets.

    Such application cannot be built with media queries.
    Since the widgets can have varying size, the module cannot change design by breakpoints relative to the viewport.
    To overcome this problem we must change the application so that widgets always have the same size.
    If we also assume that all widgets always have a fixed percentage, for example 25\% of the viewport width, it is possible to make the module responsive by using media queries.

    The problem now is that we have removed the reusability of the weather module, since it may only be used in applications that grant it 25\% of the viewport width.
    Also, if we decide to add a vertical menu to the application we need to change the media queries of the module.
    In more complex applications such change might result in changing hundreds of media queries.
    Even worse, if the menu is supposed to hide on user input the responsiveness of the module breaks since the layout changes dynamically.

    Additionally, it is popular to define breakpoints relative to the font size.
    Media queries can only target the font size of the document root, limiting the functionality drastically.
    With element queries, breakpoints may be defined relative to the font size of the targeted element.

    As we can see, even with limited requirements there still are significant flaws with using media queries for responsive modules.

    \subsection{Requirements}
      \begin{itemize}
        \item Parents should decide the layout of their children, and the children should adapt their inner design accordingly.
        \item Valid language syntaxes (HTML, CSS, JS).
      \end{itemize}

    First, a solution must enable developers to change the design of an element by its parent size.
    Elements should automatically respond to changes of the parent size so that the correct design can be activated for each size.

    Second, a solution must conform to the syntax of HTML, CSS, and JS so that the compatability of tools, libraries and existing projects is retained.

\section{Why is a Native Implementation troublesome?}
  \begin{itemize}
    \item Performance issues.
    \item Cite Tab Atkins of RICG (he states that it is infeasible to standardize this).
  \end{itemize}

\section{A JavaScript Implementation}
  \begin{itemize}
    \item Why is this pragmatic? Compatability, no impact (performance, language) on apps that do not need responsive modules.
    \item Satisfies the requirements for a solution given above.
    \item Present Elq's API.
    \item Present the performance.
    \item Note drawbacks (but only drawbacks for added functionality!).
  \end{itemize}

\section{Discussion and Summary of Related Work}
  \begin{itemize}
    \item Performance, APIs, Features.
    \item The mirror functionality of Elq makes it uniquely suitable for nested modules.
  \end{itemize}

\section{Conclusions}
  \begin{itemize}
    \item Production ready.
    \item Probably no standard (or not in a long time).
  \end{itemize}

%\end{document}  % This is where a 'short' article might terminate

%ACKNOWLEDGMENTS are optional
\section{Acknowledgments}
This section is optional; it is a location for you
to acknowledge grants, funding, editing assistance and
what have you.  In the present case, for example, the
authors would like to thank Gerald Murray of ACM for
his help in codifying this \textit{Author's Guide}
and the \textbf{.cls} and \textbf{.tex} files that it describes.

%
% The following two commands are all you need in the
% initial runs of your .tex file to
% produce the bibliography for the citations in your paper.
\bibliographystyle{abbrv}
\bibliography{elq}  % elq.bib is the name of the Bibliography in this case
% You must have a proper ".bib" file
%  and remember to run:
% latex bibtex latex latex
% to resolve all references
%
% ACM needs 'a single self-contained file'!
%
%APPENDICES are optional
%\balancecolumns
% That's all folks!
\end{document}
